\documentclass[a4paper,10pt]{article}
\usepackage{paper}
\usepackage{hyperref}
%\usepackage[notref,notcite,color]{showkeys}


\def\thetitle{All-set-homogeneous spaces}
\def\theauthors{Nina Lebedeva and Anton Petrunin}

\hypersetup{colorlinks=true,
citecolor=black,
linkcolor=black,
anchorcolor=black,
filecolor=black,
menucolor=black,
urlcolor=black,
pdftitle={\thetitle},
pdfauthor={\theauthors}
}

%\overfullrule=100mm

\begin{document}

%\pagestyle{empty}\renewcommand\includegraphics[2][{}]{}


\title{\thetitle}
\author{\theauthors}
\date{}
\maketitle

\begin{abstract}
A metric space is said to be all-set-homogeneous if any of its partial isometries can be extended to a genuine isometry.
We give a classification of a certain subclass of all-set-homogeneous length spaces.
\end{abstract}

\section{Main result}

The distance between two points $x$ and $y$ in a metric space $M$ will be denoted by $|x-y|_M$.
Recall that $M$ is called \emph{length} (or \emph{geodesic}) space if any two points $x,y\in M$ can be connected by a path $\gamma$ 
such that $|x-y|_M$ is arbitrarily close to the length of $\gamma$ (or $|x-y|_M=\length\gamma$ respectively).
Evidently, any geodesic space is a length space, but not the other way around.

A metric space $M$ is said to be \emph{all-set-homogeneous} if for any subset $A\subset M$ any distance-preserving map $A\to M$ can be extended to an isometry $M\to M$.

Examples of geodesic all-set-homogeneous spaces include complete simply-connected Rieamnnian manifolds with constant curvature and the circle equipped with length metrics.
These will be referred further as \emph{classical spaces};
they are closely related to classical Euclidean/non-Euclidean geometry.

It is worth mentioning that an infinite-dimensional Hilbert space is \emph{not} all-set-homogeneous;
indeed, it is isometric to its proper subset.
Also, for $n\ge2$, the real projective space $\RP^n$ with canonical metric is not all-set-homogeneous.
Indeed, it contains two isometric but noncongruent triples of points with pairwise distance $\tfrac\pi3$
(we assume that a closed geodesic on $\RP^n$ has length $\pi$);
one triple lies on a closed geodesic and another does not. 

Nonclassical examples include the universal metric trees of finite valence;
these are discussed in the next section.

Given a metric space $M$ and a positive integer $n$, consider all pseudometrics induced on $n$ points $x_1,\dots, x_n\in M$.
Any such metric is completely described by $N=\tfrac{n\cdot (n-1)}2$ distances $|x_i-x_j|_M$ for $i<j$, so it can be encoded by a point in $\RR^N$.
The set of all these points $F_n(M)\subset \RR^N$ will be called \emph{$n^\text{th}$ fingerprint} of~$M$.

\begin{thm}{Theorem}\label{all-sets}
Let $M$ be a complete all-set-homogeneous length space.
Suppose that all fingerprints of $M$ are closed.
Then $M$ is classical.
\end{thm}

The following two results are closely related to our theorem.
\begin{itemize}
\item \emph{Any complete all-set-homogeneous geodesic space with locally unique nonbifurcating geodesics is classical;} it was proved by Garrett Birkhoff \cite{birkhoff}.
\item \emph{Any locally compact three-point-homogeneous length space is classical.}
 This result was proved by Herbert Busemann \cite{busemann}; it also follows from the more general result of Jacques Tits \cite{tits} about two-point-homogeneous spaces.
\end{itemize}
For more related results, see the survey by Semeon Bogatyi \cite{bogaty} and the references therein.

\parit{Proof.}
If $M$ is locally compact, then the statement follows from the Busemann--Tits result stated above.

Assume $M$ is not locally compact.
Then there is an infinite sequence of points $x_1,x_2,\dots$ such that 
$\eps<|x_i-x_j|_M<1$ for some $\eps>0$.
Applying the Ramsey theorem, we get that for arbitrary positive integer $n$ there is a sequence $x_1,x_2,\dots,x_n$ such that all the distances
$|x_i-x_j|_M$ lie in arbitrarily small subinterval of $(\eps,1)$.
Since the fingerprints are closed, there is an arbitrarily long sequence 
$x_1,x_2,\dots,x_n$ such that 
$|x_i\z-x_j|_M= r$ for some fixed $r>0$.

Choose a maximal (with respect to inclusion) set of points $A$ with distance $r$ between any pair.
Since $M$ is all-set-homogeneous, we get that $A$ has to be infinite.
In particular, there is a map $f\:A\to A$ that is injective, but not surjective.

Note that $f$ is distance-preserving.
Since $A$ is maximal, for any $y\notin A$ we have that $|y-x|_M\ne r$ for some $x\in A$.
It follows that a distance-preserving map $M\to M$ that agrees with $f$ cannot have points of $A\setminus f(A)$ in its image.
In particular, no isometry $M\to M$ agrees with the map $f$ --- a contradiction.
\qeds

\section{Example}

Recall that geodesic space $T$ is called a \emph{metric tree} if any pair of points $x,y\in T$ are connected by a unique geodesic $[xy]_T$,
and the union of any two geodesics $[xy]_T$, and $[yz]_T$ contains the geodesic $[xz]_T$.

It is known that for any cardinality $n\ge 2$, there is a space $\TT_n$ that satisfies the following properties:
\begin{itemize}
\item The space $\TT_n$ is a complete metric tree with valence $n$ at any point.
\item $\TT_n$ is homogeneous; that is, the group of isometries acts transitively on~$\TT_n$. 
\end{itemize}
Moreover, this space is uniquely defined up to isometry and $n$-universal; the latter means that $\TT_n$ includes an isometric copy of any metric tree of maximal valence at most $n$.

The space $\TT_n$ is called
a \emph{universal metric tree of valence $n$}.
An explicit construction of $\TT_n$ is given by Anna Dyubina and Iosif Polterovich~\cite{dyubina-polterovich}.
Their proof of the universality of $\TT_n$ admits a straightforward modification that proves the following claim. 

\begin{thm}{Claim}
If $n$ is finite, then $\TT_n$ is all-set-homogeneous.
\end{thm}

Note that the claim implies that the condition on fingerprints in the theorem is necessary.
In fact, if $n\ge 3$, then the $(n+1)^{\text{th}}$ fingerprint of $\TT_n$ is not closed --- $\TT_n$ does not contain $n+1$ points on distance 1 from each other,
but it contains an arbitrarily large set with pairwise distances arbitrarily close to 1.


\parit{Proof.}
Let $A\subset \TT_n$ and $x\mapsto x'$ be a distance-preserving map $A\to \TT_n$;
denote by $A'$ its image.
Let us show that it can be extended to a distance-preserving map $\TT_n\to \TT_n$.

Applying the Zorn lemma, we can assume that $A$ is maximal; that is, the domain $A$ cannot be extended by a single point.
It remains to show that $A=\TT_n$ and $A'=\TT_n$.

Note that $A$ is closed.
Further, suppose $x,y,z\in A$ and $s\in [yz]_{\TT_n}$.
Since $\TT_n$ is a metric tree, the distance $|x-s|_{\TT_n}$ is completely determined by four values $|x-y|_{\TT_n}$, $|x-z|_{\TT_n}$, $|s-y|_{\TT_n}$, $|s-z|_{\TT_n}$.

Denote by $s'$ the point on the geodesic $[y'z']_{\TT_n}$ such that $|y'-s'|_{\TT_n}\z=|y-s|_{\TT_n}$ and therefore $|z'-s'|_{\TT_n}=|z-s|_{\TT_n}$.
Since the map preserves distances $|x-y|_{\TT_n}$ and $|x-z|_{\TT_n}$, we get $|s'-x'|_{\TT_n}=|s-x|_{\TT_n}$;
that is, the extension of the map by $s\mapsto s'$ is still distance-preserving.

Since $A$ is maximal, $s\in A$.
In other words, $A$ is a convex subset of $\TT_n$;
in particular, $A$ is a metric tree with maximal valence at most~$n$.

Arguing by contradiction, suppose $A\ne \TT_n$, choose $a\in A$ and $b\notin A$. 
Let $c\in A$ be the last point on the geodesic $[ab]_{\TT_n}$.
Note that the valence of $c$ in $A$ is smaller than $n$.

Since $n$ is finite, at least one of the connected components in $\TT_n\z\setminus \{c'\}$ does not intersect $A'$.
Choose a point $b'$ in this component such that $|c'-b'|_{\TT_n}=|c-b|_{\TT_n}$.
Observe that the map can be extended by $b\mapsto b'$ --- a contradiction.
It follows that $A=\TT_n$.

It remains to show that $A'=\TT_n$.
Note that $A'$ is a closed convex set in $\TT_n$ that is isometric to $\TT_n$.
Assume $A'$ is a proper subset of $\TT_n$.
Choose $a'\in A'$ and $b'\notin A'$.
Let $c'\in A'$ be the last point on the geodesic $[a'b']_{\TT_n}$.
Observe that the valence of $c'$ in $A'$ is smaller than $n$ --- a contradiction.
\qeds

\section{Remarks}


Let us list examples for related classification problems.
We would like to see any other example or a proof of the corresponding classification. 

First of all, we do not see other examples of complete all-set-homogeneous length spaces except those listed in the theorem and the claim.

Without length-metric assumption, we have a vast amount of examples.
It includes finite discrete spaces, Cantor sets with natural ultrametrics;
also note that snowflaking $(X,|\ -\ |^\theta)$ of any all-set-homogeneous spaces $(X,|\ -\ |)$ is all-set-homogeneous.

The definition of all-set-homogeneous spaces can be restricted to the distance-preserving map with \emph{small} domains; for example, \emph{finite} or \emph{compact} domains.
In these cases, we say that the space is \emph{finite-set-homogeneous} or \emph{compact-set-homogeneous} respectively.

Examples of complete separable compact-set-homogeneous length spaces include the spaces listed in the theorem,
plus the Urysohn spaces $\UU$ and $\UU_d$ (the space $\UU_d$ is isometric to a sphere of radius $\tfrac d2$ in $\UU$).
Without the separability condition, we get in addition the metric trees from the claim.

The finite-set-homogeneous spaces include, in addition, infinite-dimensional analogs of the classical spaces;
in particular the Hilbert space.  

Let us also mention that finite-set homogeneity is closely related to the \emph{metric version of Fraïssé limit} introduced by Itay Ben-Yaacov \cite{ben-yaacov}. 

\parbf{Acknowledgments.}
This note is inspired by the question of Joseph O'Rourke \cite{rourke}.
We want to thank James Hanson for his interesting and detailed comments to our question \cite{hanson}.
The second author wants to thank Rostislav Matveyev for an interesting discussion on Rubinstein Street. 

The first author was partially supported by the Russian Foundation for Basic Research grant 20-01-00070; the second author was partially supported by the National Science Foundation grant DMS-2005279.

{\sloppy
\printbibliography[heading=bibintoc]
\fussy
}

\end{document}


\parit{Proof of lemma.}
Let $M$ be a locally compact 2-point homogeneous complete length space;
denote by $G$ its group of isometries.

By the Hopf--Rinow theorem, $M$ is \emph{proper};
that is a any closed bounded set in $M$ is compact.

Let us choose a point $p\in M$.
The topology on $G$ is defined by a sequence of pseudometrics $|\ -\ |_R$ on $G$ defined by
\[|g-h|_R=\sup_{x\in \bar B[p,R]_M} \{|g(x)-h(x)|_M\}.\]
Since the closed balls $\bar B[p,R]_M$ are compact, we get that $G$ is locally compact.

Suppose $G$ is not a Lie group.
By Gleason--Yamabe theorem, for any $\eps>0$ there is a small group $K$ ... is a normal subgroup in an open subgroup $G'<G$.

Note that orbit $K\cdot x$ is nowhere dense in $G$; indeed...

Further note that the open subgroup $G'$ has at most countable number of cosets in $G$.
Therefore $G$ contains at most countable set $\{K_\alpha\}$ of subgroups that are conjugate to $K$.

According to Bair theorem the union $X=\bigcup_\alpha K_\alpha\cdot x$ is nowhere dense in $M$.
Therefore, given $x\in M$ we can choose $y,z\in M$ such that $|x-y|_M=|x-z|_M$, $y\in X$ and $z\notin X$.
Observe that there is no element $g\in G$ such that $g(x)=x$ and $g(y)=z$ --- a contradiction.
\qeds

