\documentclass[a4paper,10pt]{article}
\usepackage{paper}
\usepackage{hyperref}
%\usepackage[notref,notcite,color]{showkeys}


\def\thetitle{All-set-homogeneous spaces}
\def\theauthors{Nina Lebedeva and Anton Petrunin}

\hypersetup{colorlinks=true,
citecolor=black,
linkcolor=black,
anchorcolor=black,
filecolor=black,
menucolor=black,
urlcolor=black,
pdftitle={\thetitle},
pdfauthor={\theauthors}
}

%\overfullrule=100mm

\begin{document}

%\pagestyle{empty}\renewcommand\includegraphics[2][{}]{}


\title{\thetitle}
\author{\theauthors}
\date{}
\maketitle

\begin{abstract}
A metric space is said to be all-set-homogeneous if any of its partial isometries can be extended to a genuine isometry.
We give a classification of a certain subclass of all-set-homogeneous length spaces.
\end{abstract}

\section{Main result}

A metric space $M$ is said to be \emph{all-set-homogeneous} if for any subset $A\subset M$ any distance-preserving map $A\to M$ can be extended to an isometry $M\to M$.

Examples of all-set-homogeneous spaces include all \emph{classical spaces};
these are complete simply-connected Rieamnnian maniflds.

Nonclassical examples include the universal $\RR$-trees of finite valence;
these are discussed in the next section.

The following two results are closely related to our theorem; see also the survey by Semeon Bogatyi \cite{bogaty}.
\begin{itemize}
\item \emph{Any complete all-set-homogeneous geodesic space with locally unique nonbifurcating geodesics is classical;} it was proved by Garrett Birkhoff \cite{birkhoff}.
\item \emph{Any locally compact three-point-homogeneous geodesic space is classical.}
 This result was proved by Herbert Busemann \cite{busemann}; it also follows from the more general result of Jacques Tits \cite{tits} about two-point-homogeneous spaces.
\end{itemize}


Given a metric space $M$ and a positive integer $n$, consider all pseudometrics induced on $n$ points $x_1,\dots, x_n\in M$.
Any such metric is completely described by $N=\tfrac{n\cdot (n-1)}2$ distances $|x_i-x_j|_M$ for $i<j$, so it can be encoded by a point in $\RR^N$.
The set of all these points $F_n(M)\subset \RR^N$ will be called \emph{$n^\text{th}$ fingerprint} of~$M$.

\begin{thm}{Theorem}\label{all-sets}
Let $M$ be a complete all-set-homogeneous length space.
Suppose that all fingerprints of $M$ are closed.
Then $M$ is classical.
\end{thm}

\parit{Proof.}
If $M$ is locally compact, then the statement follows from the result of Jacques Tits \cite{tits}.

Assume $M$ is not locally compact.
Then there is an infinite sequence of points $x_1,x_2,\dots$ such that 
$\eps<|x_i-x_j|<1$ for some $\eps>0$.
Applying the Ramsey theorem, we get that for arbitrary positive integer $n$ and $\delta>0$ there is a sequence $x_1,x_2,\dots,x_n$ such that 
$|x_i-x_j|\lessgtr r\pm \delta$ where $\eps \le r\le 1$.
Since the fingerprints are closed, there is an arbitrarily long sequence 
$x_1,x_2,\dots,x_n$ such that 
$|x_i\z-x_j|= r$ for some fixed $r>0$.

Choose a maximal (with respect to inclusion) set of points $\{x_\alpha\}_{\alpha\in \mathcal{A}}$ such that $|x_\alpha-x_\beta|=r$ for any $\alpha\ne \beta$.
Since $M$ is all-set-homogeneous, we can assume that $\mathcal{A}$ is infinite.
In particular, there is an injective map $f\:\mathcal{A}\to\mathcal{A}$ such that $f(\mathcal{A})$ is a proper subset of $\mathcal{A}$.

Note that the map $x_\alpha\mapsto x_{f(\alpha)}$ is distance preserving.
Since $\{x_\alpha\}_{\alpha\in \mathcal{A}}$ is maximal, for any $y\notin \{x_\alpha\}_{\alpha\in \mathcal{A}}$ we have that $|y-x_\alpha|_M\ne r$ for some $\alpha$.
It follows that a distance preserving map $M\to M$ that agrees with $x_\alpha\mapsto x_{f(\alpha)}$ cannot have in its image a point $x_\alpha$ for $\alpha\in \mathcal{A}\setminus f(\mathcal{A})$.
In particular, no isometry $M\to M$ agrees with the map $x_\alpha\mapsto x_{f(\alpha)}$ --- a contradiction.
\qeds

\section{Example}

For any cardinality $n\ge 2$ there is a uniquely defined up to isometry space $\TT_n$ that satisfies the following properties:
\begin{itemize}
\item The space $\TT_n$ is a complete $\RR$-tree; in particular, $\TT_n$ is geodesic.
\item $\TT_n$ is homogeneous; that is, the group of isometries acts transitively on~$\TT_n$.
\item The space $\TT_n$ is $n$-universal; that is, $\TT_n$ includes an isometric copy of any $\RR$-tree of maximal valence at most $n$. 
\end{itemize}

The space $\TT_n$ is called
a \emph{universal $\RR$-tree of valence $n$}.
An explicit construction of $\TT_n$ is given by Anna Dyubina and Iosif Polterovich~\cite{dyubina-polterovich}.
Their proof of the universality of $\TT_n$ admits a straightforward modification that proves the following claim. 

\begin{thm}{Claim}
If $n$ is finite, then $\TT_n$ is all-set-homogeneous.
\end{thm}

Note that the claim implies that the condition on fingerprints in the theorem is necessary.
In fact, if $n\ge 3$, then the $(n+1)^{\text{th}}$ fingerprint of $\TT_n$ is not closed --- $\TT_n$ does not contain $n+1$ points on distance 1 from each other,
but it contains an arbitrarily large set with pairwise distances arbitrarily close to 1.


\parit{Proof.}
Let $f\:A\to \TT_n$ be a distance preserving map for some subset $A\subset \TT_n$.
Let us extend $f$ to a distance preserving map $\TT_n\to \TT_n$.

Applying the Zorn lemma, we can assume that $A$ is maximal; that is, the domain of $f$ cannot be extended by a single point.
Note that in this case, $A$ is a closed convex set in $\TT_n$; in particular, $A$ is an $\RR$-tree with maximal valence at most $n$.

Arguing by contradiction, suppose $A\ne \TT_n$, choose $a\in A$ and $b\notin A$. 
Let $c\in A$ be the last point on the geodesic $[ab]_{\TT_n}$.
Note that the valence of $c$ in $A$ is smaller than $n$.

Let $c'=f(c)$; since $n$ is finite, at least one of connected components of $\TT_n\z\setminus \{c'\}$ does not intersect $A'=f(A)$.
Choose a point $b'$ in this component such that $|c'-b'|_{\TT_n}=|c-b|_{\TT_n}$.
Observe that $f$ can be extended by $b\mapsto b'$ --- a contradiction.

It remains to show that $f(\TT_n)=\TT_n$ for any distance-preserving map $f\:\TT_n\z\to \TT_n$.
Assume the contrary; that is, $B=f(\TT_n)$ is a proper subset on $\TT_n$.
Note that $B$ is a closed convex set in $\TT_n$.
Choose $a\in B$ and $b\not \in B$.
Let $c\in B$ be the last point on the geodesic $[ab]_{\TT_n}$.
Observe that the valence of $c$ in $B$ is smaller than $n$ --- a contradiction.
\qeds

\section{Remarks}


Let us list examples for related classification problems.
We would be interested to see other examples or a proof that there are no more. 

First of all, we do not see other examples of complete all-set-homogeneous length spaces except those listed in the theorem and the claim.

Without length-metric assumption, examples include finite discrete spaces, Cantor sets with natural ultrametrics, and many more spaces.

The definition of all-set-homogeneous spaces can be restricted to the distance-preserving map with \emph{small} domains; for example, \emph{finite} or \emph{compact} domains.
In these cases, we say that the space is \emph{finite-set-homogeneous} or \emph{compact-set-homogeneous} respectively.

Examples of complete separable compact-set-homogeneous length spaces include the spaces listed in the theorem,
plus the Urysohn spaces $\UU$ and $\UU_d$ (the space $\UU_d$ is isometric to a sphere of radius $\tfrac d2$ in $\UU$).
Without the separability condition, we get in addition the $\RR$-trees from the claim.

The finite-set-homogeneous spaces include, in addition, infinite-dimensional analogs of the spaces in the theorem;
in particular the Hilbert space.  

Let us also mention that finite-set homogeneity is closely related to the \emph{metric version of Fraïssé limit} introduced by Itay Ben-Yaacov \cite{ben-yaacov}. 

\parbf{Acknowledgments.}
This note is inspired by the question of Joseph O'Rourke \cite{rourke}.
We want to thank James Hanson for his interesting and detailed comments to our question \cite{hanson}.
The second author wants to thank Rostislav Matveyev for an interesting discussion on Rubinstein Street. 

The first author was partially supported by the Russian Foundation for Basic Research grant 20-01-00070, the second author was partially supported by the National Science Foundation grant DMS-2005279.

{\sloppy
\printbibliography[heading=bibintoc]
\fussy
}

\end{document}


\parit{Proof of lemma.}
Let $M$ be a locally compact 2-point homogeneous complete length space;
denote by $G$ its group of isometries.

By the Hopf--Rinow theorem, $M$ is \emph{proper};
that is a any closed bounded set in $M$ is compact.

Let us choose a point $p\in M$.
The topology on $G$ is defined by a sequence of pseudometrics $|\ -\ |_R$ on $G$ defined by
\[|g-h|_R=\sup_{x\in \bar B[p,R]_M} \{|g(x)-h(x)|_M\}.\]
Since the closed balls $\bar B[p,R]_M$ are compact, we get that $G$ is locally compact.

Suppose $G$ is not a Lie group.
By Gleason--Yamabe theorem, for any $\eps>0$ there is a small group $K$ ... is a normal subgroup in an open subgroup $G'<G$.

Note that orbit $K\cdot x$ is nowhere dense in $G$; indeed...

Further note that the open subgroup $G'$ has at most countable number of cosets in $G$.
Therefore $G$ contains at most countable set $\{K_\alpha\}$ of subgroups that are conjugate to $K$.

According to Bair theorem the union $X=\bigcup_\alpha K_\alpha\cdot x$ is nowhere dense in $M$.
Therefore, given $x\in M$ we can choose $y,z\in M$ such that $|x-y|_M=|x-z|_M$, $y\in X$ and $z\notin X$.
Observe that there is no element $g\in G$ such that $g(x)=x$ and $g(y)=z$ --- a contradiction.
\qeds

