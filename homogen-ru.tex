\documentclass[a4paper,10pt]{article}
\usepackage{paper-ru}
\usepackage{hyperref}
%\usepackage[notref,notcite,color]{showkeys}


\def\thetitle{Полностью однородные пространства}
\def\theauthors{Н. Д. Лебедева и А. М. Петрунин}

\hypersetup{colorlinks=true,
citecolor=black,
linkcolor=black,
anchorcolor=black,
filecolor=black,
menucolor=black,
urlcolor=black,
pdftitle={\thetitle},
pdfauthor={\theauthors}
}

%\overfullrule=100mm

\begin{document}

%\pagestyle{empty}\renewcommand\includegraphics[2][{}]{}


\title{\thetitle}
\author{\theauthors}
\date{}
\maketitle

\begin{abstract}
Метрическое пространство называется полностью однородным, если любая изометрия между его подмножествами продолжается до изометрии всего пространства.
Мы классифицируем определённый подкласс полностью однородных пространств с внутренней метрикой.
\end{abstract}

\section{Основной результат}

Расстояние между точками $x$ и $y$ метрического пространства $M$ будет обозначаться как $|x-y|_M$.
Метрика на $M$ называется \emph{внутренней} (\emph{геодезической}) если любые две точки $x,y\in M$ можно соединить путём $\gamma$ таким, что расстояние $|x-y|_M$ произвольно близко к длине $\gamma$ (или, соответственно, равно этой длине).
Очевидно, что любая геодезическая метрика внутренняя, но не наоборот.

Метрическое пространство $M$ называется \emph{полностью однородным}, если любая изометрия $A\to A'$ между его подмножествами продолжается до изометрии $M\to M$ всего пространства.

Примеры геодезических полностью однородных пространств включают односвязные римановы многообразия постоянной кривизны и окружности с внутренней метрикой.
Эти примеры будут называться \emph{классическими пространствами},
они тесно связанны с классической евклидовой/неевклидовой геометрией.

Стоит заметить, что бесконечномерное гильбертово пространство не полностью однородно --- это следует из того, что оно изометрично своему собственному подмножеству.
Далее, при $n\ge2$, вещественное проективное пространство $\RP^n$ с канонической метрикой не полностью однородно.
Действительно, $\RP^n$ содержит две изометричные, но не конгруэнтные тройки точек с попарными расстояниями $\tfrac\pi3$ ---
одна тройка лежит на замкнутой геодезической, а другая нет. 
(Мы предполагаем, что длина замкнутой геодезической на $\RP^n$ равна $\pi$.)

Неклассические примеры включают универсальные деревья конечной валентности, они обсуждаются в следующем разделе.

Для метрического пространства $M$ и целого числа $n\ge1$ рассмотрим все псевдометрики на $n$-ках точек 
$x_1,\dots, x_n\in M$.
Каждая такая метрика описывается $N=\tfrac{n\cdot (n-1)}2$ расстояниями $|x_i-x_j|_M$ для $i<j$, и, таким образом, соответствует точке в $\RR^N$.
Множество всех таких точек $F_n(M)\z\subset \RR^N$ будет называться $n$-м отпечатком $M$.

\begin{thm}{Теорема}\label{all-sets}
Пусть $M$ --- полное полностью однородное пространство с внутренней метрикой.
Предположим, что все отпечатки $M$ замкнуты.
Тогда $M$ --- классическое пространство.
\end{thm}

Следующие два утверждения очень близки нашему результату:
\begin{itemize}
\item \emph{Любое полностью однородное геодезическое пространство с локально однозначными не расщепляющимися геодезическими является классическим.} Доказано Гарретом Биркгофом  \cite{birkhoff}.
\item \emph{Любое локально компактное 3-точечно однородное пространство с внутренней метрикой  является классическим.} 
Доказано Гербертом Буземаном \cite{busemann}.
Это утверждение также следует из более общего результата Жака Титса \cite{tits} о двуточечно однородных пространствах.
(Пространство называется $n$-точечно однородным, если любая изометрия между его подмножествами с не более чем $n$ точками в каждом продолжается до изометрии всего пространства.)
\end{itemize}
Подробное обсуждение этих и других связанных результатов дано в обзоре Семёна Богатого \cite{bogaty-ru}.

\parit{Доказательство.}
Если $M$ локально компактно, то утверждение следует из приведённой выше теоремы Буземана --- Титса.
Поэтому можно предположить, что $M$ не локально компактно.

В этом случае существует бесконечная последовательность точек $x_1$, $x_2,\dots$ такая, что $\eps<|x_i-x_j|_M<1$ для некоторого $\eps>0$ и всех $i\ne j$.
Применив теорему Рамсея, получаем, что для произвольного целого $n\ge 1$ существует последовательность $x_1,x_2,\dots,x_n$ такая, что все расстояния $|x_i-x_j|_M$ лежат в произвольно малом подинтервале $(\eps,1)$.
Поскольку отпечатки замкнуты, существует произвольно длинная последовательность $x_1,x_2,\dots,x_n$ такая, что
$|x_i\z-x_j|_M= r$ для некоторого фиксированного $r>0$ и всех $i\ne j$.

Выберем максимальное (относительно включения) множество $A$ с расстоянием $r$ между любыми двумя его точками.
Поскольку $M$ полностью однородно, множество $A$ обязано быть бесконечным.
В частности, существует отображение $f\:A\to A$, которое инъективно, но не сюрьективно.

Заметим, что $f$ сохраняет расстояния.
Поскольку $A$ максимально, для любой точки $y\notin A$ существует $x\in A$ такая, что $|y-x|_M\ne r$.
Значит любое отображение $M\to M$, которое согласуется с $f$ и сохраняет расстояния, не может содержать в своём образе точек из $A\setminus f(A)$.
В частности, никакая изометрия $M\to M$ не согласуется с $f$ --- противоречие.
\qeds

\section{Пример}

Напомним, что пространство $T$ с геодезической метрикой называется \emph{метрическим деревом} если любая пара точек $x,y\in T$ соединяется единственной кратчайшей $[xy]_T$ и объединение любой пары кратчайших $[xy]_T$ и $[yz]_T$ содержит $[xz]_T$.
\emph{Валентностью} точки $x\in T$ называется кардинальное число связных компонент в $T\setminus\{x\}$.

Для любого кардинального числа $n\ge 2$ сущестует пространство $\TT_n$ удовлетворяющее следующим условиям:
\begin{itemize}
\item $\TT_n$ --- полное метрическое дерево с валентностью $n$ в каждой точке.
\item $\TT_n$ --- однородно, то есть, группа его изометрий действует транзитивно. 
\end{itemize}
Более того, $\TT_n$ однозначно определено с точностью до изометрии и является $n$-универсальным.
Последнее означает, что $\TT_n$ включает в себя изометрическую копию любого метрического дерева с максимальной валентностью не больше $n$.

Пространство $\TT_n$ называется \emph{универсальным метрическим деревом валентности $n$}.
Явное построение $\TT_n$ дано Анной Дюбиной и Иосифом Полтеровичем~\cite{dyubina-polterovich}.
Доказательство следующего утверждения --- простая переделка их доказательства универсальности $\TT_n$.

\begin{thm}{Утверждение}
Если $n$ конечно, то $\TT_n$ полностью однородно.
\end{thm}

В частности, условие на отпечатки в нашей теореме является существенным.
Заметим, что если $n\ge 3$, то $(n+1)$-й отпечаток $\TT_n$ не замкнут --- $\TT_n$ не содержит $(n+1)$-точечное множество с попарными расстояниями 1, 
но содержит произвольно большое конечное множество с  попарными расстояниями произвольно близкими к 1.


\parit{Доказательство.}
Пусть $A, A'\subset \TT_n$, и $x\mapsto x'$ есть изометрия $A\to A'$.
По лемме Цорна, можно предположить, что $A$ максимально,
то есть, к $A$ невозможно добавить точку.
Остаётся доказать, что $A=\TT_n$ и $A'=\TT_n$.

Заметим, что $A$ замкнуто.

\begin{wrapfigure}{o}{23 mm}
\vskip-6mm
\centering
\includegraphics{mppics/pic-10}
\end{wrapfigure}

Далее, предположим $x,y,z\in A$ и $s\in [yz]_{\TT_n}$.
Поскольку $\TT_n$ --- метрическое дерево, расстояние $|x-s|_{\TT_n}$ полностью определяется четвёркой чисел 
$|x-y|_{\TT_n}$, $|x-z|_{\TT_n}$, $|s-y|_{\TT_n}$, $|s-z|_{\TT_n}$.

Обозначим через $s'$ точку на кратчайшей $[y'z']_{\TT_n}$ такую, что $|y'-s'|_{\TT_n}\z=|y-s|_{\TT_n}$, и значит $|z'-s'|_{\TT_n}=|z-s|_{\TT_n}$.
Поскольку отображение сохраняет расстояния $|x-y|_{\TT_n}$ и $|x-z|_{\TT_n}$,
получаем, что $|s'-x'|_{\TT_n}=|s-x|_{\TT_n}$.
То есть отображения продолженное как $s\mapsto s'$ остаётся сохраняющим расстояния.

Поскольку $A$ максимально, $s\in A$.
Другими словами, $A$ --- выпуклое подмножество в $\TT_n$.
В частности, $A$ --- метрическое дерево максимальной валентности не больше $n$.

Предположим $A\ne \TT_n$.
Выберем $a\in A$ и $b\notin A$.
Пусть $c\in A$ --- последняя точка на кратчайшей $[ab]_{\TT_n}$.
Заметим, что валентность $c$ в $A$ строго меньше $n$.

Поскольку $n$ конечно, хотя бы одна из компонент связности $\TT_n\z\setminus \{c'\}$ не пересекает $A'$.
Выберем точку $b'$ в этой компоненте так, чтобы $|c'-b'|_{\TT_n}\z=|c-b|_{\TT_n}$.
Заметим, что изометрию $A\to A'$ можно продолжить как $b\mapsto b'$ --- противоречие.
Значит $A=\TT_n$.

Осталось показать, что $A'=\TT_n$.
Заметим, что $A'$ замкнутое подмножество $\TT_n$ и оно изометрично $\TT_n$.
В частности валентность любой точки $A'$ равна $n$.

Предположим $A'$ --- собственное подмножество $\TT_n$.
Выберем $a'\in A'$ и $b'\notin A'$.
Пусть $c'\in A'$ --- последняя точка на кратчайшей $[a'b']_{\TT_n}$.
Тогда валентность $c'$ в $A'$ меньше $n$ --- противоречие.
\qeds

\section{Замечания}

Приведём известные нам примеры пространств для похожих классификационных задач.
Нас интересуют любые другие примеры а также доказательства таких классификаций.

Во-первых, мы не знаем примеров полных полностью однородных пространств с внутренней метрикой кроме классических пространств и универсальных метрических деревьев конечной валентности.

Без условия внутренности метрики, дополнительных примеров становится слишком много.
Они включают в себя конечные дискретные пространства, канторовы множества с естественными ультраметриками.
Также заметим, что \emph{снежинка} $(X,|\ -\ |^\theta)$ любого полностью однородного пространства $(X,|\ -\ |)$ полностью однородна.

В определении полностью однородных пространств можно рассматривать только маленькие подмножества $A$ и $A'$,
например, \emph{конечные} или \emph{компактные}.
Тогда мы говорим, что пространство  \emph{конечно однородно} или  \emph{компактно однородно} соответственно.

В дополнение к классическим пространствам,
примеры полных сепарабельных компактно однородных пространств с внутренней метрикой включают пространства Урысона  $\UU$ и $\UU_d$ (пространство $\UU_d$ изометрично сфере радиуса $\tfrac d2$ в $\UU$).
Без сепарабельности к ним добавляются универсальные метрические деревья конечной валентности.

Конечно однородные пространства включают дополнительно бесконечномерные аналоги классических пространств,
в частности гильбертово пространство.

Заметим также, что конечно однородные пространства тесно связаны с \emph{метрическим пределом Фраисэ} предложенным Итайем Бен Яаковом \cite{ben-yaacov}. 

\parbf{Благодарности.}
Эта заметка появилась благодаря вопросу Джозефа Орурка \cite{rourke}.
Мы хотим поблагодарить Джеймса Хансона за его интересные и подробные замечания к нашему вопросу \cite{hanson}.
Второй автор признателен Ростиславу Матвееву за интересный разговор на улице Рубинштейна.

Первый автор частично поддержан грантом РФФИ 20-01-00070;
второй автор частично поддержан грантом NSF DMS-2005279.

{\sloppy
\printbibliography[heading=bibintoc]
\fussy
}

\end{document}


\parit{Proof of lemma.}
Let $M$ be a locally compact 2-point homogeneous complete length space;
denote by $G$ its group of isometries.

By the Hopf--Rinow theorem, $M$ is \emph{proper};
that is a any closed bounded set in $M$ is compact.

Let us choose a point $p\in M$.
The topology on $G$ is defined by a sequence of pseudometrics $|\ -\ |_R$ on $G$ defined by
\[|g-h|_R=\sup_{x\in \bar B[p,R]_M} \{|g(x)-h(x)|_M\}.\]
Since the closed balls $\bar B[p,R]_M$ are compact, we get that $G$ is locally compact.

Suppose $G$ is not a Lie group.
By Gleason--Yamabe theorem, for any $\eps>0$ there is a small group $K$ ... is a normal subgroup in an open subgroup $G'<G$.

Note that orbit $K\cdot x$ is nowhere dense in $G$; indeed...

Further note that the open subgroup $G'$ has at most countable number of cosets in $G$.
Therefore $G$ contains at most countable set $\{K_\alpha\}$ of subgroups that are conjugate to $K$.

According to Bair theorem the union $X=\bigcup_\alpha K_\alpha\cdot x$ is nowhere dense in $M$.
Therefore, given $x\in M$ we can choose $y,z\in M$ such that $|x-y|_M=|x-z|_M$, $y\in X$ and $z\notin X$.
Observe that there is no element $g\in G$ such that $g(x)=x$ and $g(y)=z$ --- a contradiction.
\qeds

