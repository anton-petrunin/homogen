 \documentclass[a4paper,10pt]{article}
\usepackage{paper}
\usepackage{hyperref}
%\usepackage[notref,notcite,color]{showkeys}


\def\thetitle{All-set-homogeneous spaces}
\def\theauthors{Nina Lebedeva and Anton Petrunin}

\hypersetup{colorlinks=true,
citecolor=black,
linkcolor=black,
anchorcolor=black,
filecolor=black,
menucolor=black,
urlcolor=black
}

\begin{document}

%\pagestyle{empty}\renewcommand\includegraphics[2][{}]{}


\title{\thetitle}
\author{\theauthors}
\date{}
\maketitle

\begin{abstract}
We give a classification of a certain subclass of all-set-homogeneous length spaces;
a metric space is all-set-homogeneous if any of its partial isometries can be extended to a genuine isometry.
\end{abstract}

\section{Main result}

A metric space $M$ is said to be \emph{all-set-homogeneous} if for any subset $A\subset M$ any distance-preserving map $A\to M$ can be extended to an isometry $M\to M$.

Given a metric space $M$ and a positive integer $n$, consider all pseudometrics induced on $n$ points $x_1,\dots, x_n\in M$.
Any such metric is completely described by $N=\tfrac{n\cdot (n-1)}2$ real numbers $|x_i-x_j|$ for $i<j$, so it can be encoded by a point in $\RR^N$.
The set of all these points $F_n(M)\subset \RR^N$ will be called \emph{$n^\text{th}$ fingerprint} of~$M$.

Let us denote by $\HH^m$ the hyperbolic space,
$\EE^m$ --- Euclidean space, $\S^m$ --- unit sphere, $\RP^m$ --- projective space, all with standard metrics;
here $m$ stands for the dimension of the space.

Given a metric space $M$, we denote by $\lambda\cdot M$ the rescaled $M$ with factor $\lambda>0$.

\begin{thm}{Theorem}\label{all-sets}
Let $M$ be a complete all-set-homogeneous length space.
Suppose that all fingerprints of $M$ are closed.
Then $M$ is isometric to $\EE^m$, $\lambda\cdot \HH^m$, $\lambda\cdot\S^m$, or $\lambda\cdot\RP^m$ for some $\lambda>0$ and nonnegative integer $m$.
\end{thm}

The question about classifying all-set-homogeneous space was posted by Joseph O'Rourke \cite{rourke}.
The locally compact case of the theorem is proven by Jacques Tits \cite{tits}.
In this case, the condition on fingerprints is not needed.
Moreover, the three-point-homogeneity is sufficient; that is, any distance preserving map defined on a subset with at most 3 points can be extended to an isometry.
Our result shows that one can trade local compactness to a better type of homogeneity and closed fingerprints.

\parit{Proof.}
If $M$ is locally compact, then the statement follows from the result of Jacques Tits \cite{tits}.

Assume $M$ is not locally compact.
Then there is an infinite sequence of points $x_1,x_2,\dots$ such that 
$\eps<|x_i-x_j|<1$ for some $\eps>0$.
Applying the Ramsey theorem, we get that for arbitrary positive integer $n$ and $\delta>0$ there is a sequence $x_1,x_2,\dots,x_n$ such that 
$|x_i-x_j|\lessgtr r\pm \delta$ where $\eps \le r\le 1$.
Since the fingerprints are closed, there is an arbitrarily long sequence 
$x_1,x_2,\dots,x_n$ such that 
$|x_i\z-x_j|= r$ for some fixed $r>0$.

Choose a maximal (with respect to inclusion) set of points $\{x_\alpha\}_{\alpha\in \mathcal{A}}$ such that $|x_\alpha-x_\beta|=r$ for any $\alpha\ne \beta$.
Since $M$ is all-set-homogeneous, we can assume that $\mathcal{A}$ is infinite.
In particular, there is an injective map $f\:\mathcal{A}\to\mathcal{A}$ such that $f(\mathcal{A})$ is a proper subset of $\mathcal{A}$.

Note that the map $x_\alpha\mapsto x_{f(\alpha)}$ is distance preserving.
Since $\{x_\alpha\}_{\alpha\in \mathcal{A}}$ is maximal, for any $y\notin \{x_\alpha\}_{\alpha\in \mathcal{A}}$ we have that $|y-x_\alpha|_M\ne r$ for some $\alpha$.
It follows that a distance preserving map $M\to M$ that agrees with $x_\alpha\mapsto x_{f(\alpha)}$ cannot have in its image a point $x_\alpha$ for $\alpha\in \mathcal{A}\setminus f(\mathcal{A})$.
In particular, no isometry $M\to M$ agrees with the map $x_\alpha\mapsto x_{f(\alpha)}$ --- a contradiction.
\qeds

\section{Example}

Recall that for any cardinality $n\ge 2$ there is the so-called \emph{universal $\RR$-tree of valence $n$}; let us denote it by $\TT_n$.
\begin{itemize}
\item The space $\TT_n$ is a complete $\RR$-tree; in particular, $\TT_n$ is geodesic.
\item $\TT_n$ is homogeneous; that is, the group of isometries acts transitively on $\TT_n$.
\item The space $\TT_n$ is $n$-universal; that is, $\TT_n$ includes a copy of any $\RR$-tree of maximal valence at most $n$
\end{itemize}

An explicit construction of $\TT_n$ is given by Anna Dyubina (now Erschler) and Iosif Polterovich \cite{dyubina-polterovich}.
Their proof of the universality of $\TT_n$ admits a straightforward modification that proves the following claim. 

\begin{thm}{Claim}
If $n$ is finite, then $\TT_n$ is all-set-homogeneous.
\end{thm}

The claim implies that the condition on fingerprints in the theorem is necessary.

\parit{Proof.}
Choose a subset $A\subset \TT_n$. 
Let us extend any distance preserving map $f\:A\to \TT_n$ to a distance preserving map $\TT_n\to \TT_n$.

Applying the Zorn lemma, we can assume that $A$ is maximal; that is, the domain of $f$ cannot be extended by a single point.
Note that in this case, $A$ is a closed convex set in $\TT_n$; in particular, $A$ is an $\RR$-tree with maximal valence at most $n$.
Suppose $A\ne \TT_n$, choose $a\in A$ and $b\notin A$. 
Let $c\in A$ be the last point on the geodesic $[ab]_{\TT_n}$.
Note that the valence of $c$ in $A$ is smaller than $n$.

Let $c'=f(c)$; since $n$ is finite at least one of connected components $\TT_n\setminus \{c'\}$ does not intersect $A'=f(A)$.
Choose a point $b'$ in this connected component on distance $|c-b|$ from $c'$.
Observe that $f$ can be extended by $b\mapsto b'$ --- a contradiction.

It remains to show that $f(\TT_n)=\TT_n$ for any distance-preserving map $f\:\TT_n\z\to \TT_n$.
Assume the contrary; that is, $B=f(\TT_n)$ is a proper subset on $\TT_n$.
Choose a point $c\in \partial B$; note that the valence of $c$ in $B$ is smaller than $n$ --- a contradiction.
\qeds

\section{Remarks}


Let us list examples for related classification problems.
We would be interested to see other examples or a proof that there are no more. 

First of all, we do not see other examples of complete all-set-homogeneous length spaces except those listed in the theorem and the claim.

The definition of all-set homogeneity can be restricted to the distance-preserving map with \emph{small} domains; for example, \emph{finite} or \emph{compact} domains.
In these cases, we say that the space is \emph{finite-set-homogeneous} or \emph{compact-set-homogeneous} respectively.

Examples of complete separable compact-set-homogeneous length spaces with closed fingerprints include the spaces listed in the theorem,
plus the Urysohn space $\UU$ and its version $\UU_d$ of diameter $d$ (it is isometric to a sphere of radius $\tfrac d2$ in $\UU$).
Without separability condition, the $\RR$-trees from the claim.

For finite-set-homogeneous we get in addition infinite-dimensional analogs of the spaces in the theorem;
in particular the Hilbert space.

\parbf{Acknowledgments.}
The second author wants to thank Rostislav Matveyev for an interesting discussion on the Rubinstein Street. 

The first author was partially supported by the Russian Foundation for Basic Research grant 20-01-00070, 
the second author was partially supported by the National Science Foundation grant DMS-2005279.

{\sloppy
\printbibliography[heading=bibintoc]
\fussy
}

\end{document}


\parit{Proof of lemma.}
Let $M$ be a locally compact 2-point homogeneous complete length space;
denote by $G$ its group of isometries.

By the Hopf--Rinow theorem, $M$ is \emph{proper};
that is a any closed bounded set in $M$ is compact.

Let us choose a point $p\in M$.
The topology on $G$ is defined by a sequence of pseudometrics $|\ -\ |_R$ on $G$ defined by
\[|g-h|_R=\sup_{x\in \bar B[p,R]_M} \{|g(x)-h(x)|_M\}.\]
Since the closed balls $\bar B[p,R]_M$ are compact, we get that $G$ is locally compact.

Suppose $G$ is not a Lie group.
By Gleason--Yamabe theorem, for any $\eps>0$ there is a small group $K$ ... is a normal subgroup in an open subgroup $G'<G$.

Note that orbit $K\cdot x$ is nowhere dense in $G$; indeed...

Further note that the open subgroup $G'$ has at most countable number of cosets in $G$.
Therefore $G$ contains at most countable set $\{K_\alpha\}$ of subgroups that are conjugate to $K$.

According to Bair theorem the union $X=\bigcup_\alpha K_\alpha\cdot x$ is nowhere dense in $M$.
Therefore, given $x\in M$ we can choose $y,z\in M$ such that $|x-y|_M=|x-z|_M$, $y\in X$ and $z\notin X$.
Observe that there is no element $g\in G$ such that $g(x)=x$ and $g(y)=z$ --- a contradiction.
\qeds
